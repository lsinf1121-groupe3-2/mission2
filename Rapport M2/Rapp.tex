\documentclass[11pt]{article}
\usepackage[utf8]{inputenc}
\usepackage[T1]{fontenc}
\usepackage[final]{pdfpages} 
\usepackage[french]{babel}
\usepackage{amsmath}
\usepackage[bookmarks={true},bookmarksopen={true}]{hyperref}
\usepackage{graphicx}
\usepackage[a4paper]{geometry}
\usepackage{listings}
	\lstset{frame=tb,
		language=Java,
 		aboveskip=3mm,
  		belowskip=3mm,
  		showstringspaces=false,
  		columns=flexible,
  		basicstyle={\small\ttfamily},
  		numbers=none,
 		numberstyle=\tiny\color{gray},
  		keywordstyle=\color{blue},
  		commentstyle=\color{dkgreen},
  		stringstyle=\color{mauve},
  		breaklines=true,
  		breakatwhitespace=true
  		tabsize=3
	}
\pagestyle{plain}
\setlength{\parindent}{5mm}

\usepackage{color}

\definecolor{dkgreen}{rgb}{0,0.6,0}
\definecolor{gray}{rgb}{0.5,0.5,0.5}
\definecolor{mauve}{rgb}{0.58,0,0.82}



\title{\textbf{Projet LSINF1121 -  Algorithmique et structures de données\\ - \\ Rapport final Mission 2} \\ {\large Groupe 3.2}}
\author{Boris \bsc{Dehem} \\(5586-12-00)\and Sundeep \bsc{Dhillon} \\(6401-11-00)\and Alexandre \bsc{Hauet} \\ (5336-08-00) \and Jonathan \bsc{Powell}\\(6133-12-00)\and Mathieu \bsc{Rosar} \\ (4718-12-00)\and Tanguy \bsc{Vaessen} \\ (0810-14-00)}
\date{date}
\date{\vspace*{25mm}
\includegraphics[scale=0.75]{logo.jpg}\\
		\vspace*{30mm}
		\begin{center}
		Année académique 2014-2015 \\	
		\end{center}}

\begin{document}
\thispagestyle{empty}

\maketitle
\thispagestyle{empty}
%\tableofcontents
%\setcounter{tocdepth}{3}
%\setcounter{page}{1}
%\newpage
\section*{Introduction}
Dans le cadre du cours ``Algorithmique et structures de données'', il nous a été demandé d'implémenter un programme de dérivation formelle. Ce programme devra utiliser le type abstrait arbre pour représenter et manipuler les expressions analytique.
\section{Choix d'implémentation}
Pour la représention de l'arbre binaire nous avons décidé de réutiliser l'implémentation de RBinaryTree décrite dans DSAJ-5.

\section{Diagramme de classes}

\section{Difficultés rencontrées}
Grâce aux questions de la séance intermédiaire, nous avons décortiqué le problème ce qui nous a permis de préparer correctement le partie implémenation. Nous n'avons donc pas rencontré de réelle problème lors de la partie dévellopement de l'application. 


\section{Analyse de la complexité calculatoire}

\subsection{Complexité temporelle}
\subsubsection*{Méthodes de la classe LinkedRBinaryTree}
Les méthodes de la classe LinkedRBinaryTree sont de complexité O(1).

Sauf les méthodes suivantes :
\begin{itemize}
	\item[size]
	\item[search]
	\item[toString]
\end{itemize}



\subsection{Complexité spatiale}

\section{Répartition du travail}

\end{document}
